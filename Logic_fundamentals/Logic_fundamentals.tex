\documentclass[a4paper,12pt]{article}
\usepackage[utf8]{inputenc}
\usepackage{amsmath, amssymb, amsthm}
\usepackage{geometry}
\geometry{margin=1in}
\usepackage{enumitem}

\title{Comprehensive Guide to Logic Basics, Quantifiers, and Set Theory}
\author{}
\date{}

\begin{document}

\maketitle

\tableofcontents

\section{Introduction}
Logic, quantifiers, and set theory form the foundational pillars of mathematics and computer science. This guide provides an introduction to these concepts, their notation, and usage.

\section{Logic Basics}
Logic is the study of reasoning and principles of valid inference. It provides tools to evaluate the truth of statements.

\subsection{Propositional Logic}
Propositional logic deals with propositions, which are statements that are either true (T) or false (F).
\begin{itemize}
    \item \textbf{Logical Operators:}
    \begin{itemize}
        \item \textbf{Negation (\(\neg\)):} \(\neg P\) is true if \(P\) is false.
        \item \textbf{Conjunction (\(\land\)):} \(P \land Q\) is true if both \(P\) and \(Q\) are true.
        \item \textbf{Disjunction (\(\lor\)):} \(P \lor Q\) is true if at least one of \(P\) or \(Q\) is true.
        \item \textbf{Implication (\(\implies\)):} \(P \implies Q\) is false only if \(P\) is true and \(Q\) is false.
        \item \textbf{Biconditional (\(\iff\)):} \(P \iff Q\) is true if \(P\) and \(Q\) have the same truth value.
    \end{itemize}
\end{itemize}

\subsection{Truth Tables}
Truth tables are used to systematically explore all possible truth values of logical expressions. Example:
\[
\begin{array}{|c|c|c|}
\hline
P & Q & P \land Q \\
\hline
T & T & T \\
T & F & F \\
F & T & F \\
F & F & F \\
\hline
\end{array}
\]

\subsection{Logical Equivalence}
Two statements are logically equivalent if they have the same truth table. For example:
\[\neg(P \lor Q) \equiv \neg P \land \neg Q \text{ (De Morgan's Law)}\]

\section{Quantifiers}
Quantifiers extend logic to statements about collections of objects, often expressed as variables.

\subsection{Universal Quantifier (\(\forall\))}
\(\forall x \in S, P(x)\): The statement is true if \(P(x)\) is true for all elements \(x\) in the set \(S\).
\begin{itemize}
    \item Example: \(\forall x \in \mathbb{R}, x^2 \geq 0\).
\end{itemize}

\subsection{Existential Quantifier (\(\exists\))}
\(\exists x \in S, P(x)\): The statement is true if there exists at least one element \(x\) in \(S\) such that \(P(x)\) is true.
\begin{itemize}
    \item Example: \(\exists x \in \mathbb{R}, x^2 = 4\).
\end{itemize}

\subsection{Negating Quantifiers}
Negating statements with quantifiers involves switching between \(\forall\) and \(\exists\):
\[\neg(\forall x, P(x)) \equiv \exists x, \neg P(x)\]
\[\neg(\exists x, P(x)) \equiv \forall x, \neg P(x)\]

\section{Set Theory}
Set theory studies collections of objects, called sets. Sets are fundamental in mathematics.

\subsection{Basic Concepts}
\begin{itemize}
    \item \textbf{Set:} A collection of distinct elements. Example: \(A = \{1, 2, 3\}\).
    \item \textbf{Element:} If \(x\) is in \(A\), we write \(x \in A\).
    \item \textbf{Empty Set:} The set with no elements, denoted \(\emptyset\).
    \item \textbf{Subset:} \(A \subseteq B\) if every element of \(A\) is also in \(B\).
\end{itemize}

\subsection{Set Operations}
\begin{itemize}
    \item \textbf{Union (\(\cup\)):} \(A \cup B = \{x : x \in A \text{ or } x \in B\}\).
    \item \textbf{Intersection (\(\cap\)):} \(A \cap B = \{x : x \in A \text{ and } x \in B\}\).
    \item \textbf{Difference (\(\setminus\)):} \(A \setminus B = \{x : x \in A \text{ and } x \notin B\}\).
    \item \textbf{Complement (\(A^c\)):} \(A^c = \{x : x \notin A\}\).
\end{itemize}

\subsection{Power Set}
The power set of \(A\), denoted \(\mathcal{P}(A)\), is the set of all subsets of \(A\).
\[\mathcal{P}(\{1, 2\}) = \{\emptyset, \{1\}, \{2\}, \{1, 2\}\}\]

\subsection{Cartesian Product}
The Cartesian product of \(A\) and \(B\), denoted \(A \times B\), is the set of ordered pairs:
\[A \times B = \{(a, b) : a \in A, b \in B\}\]

\subsection{Venn Diagrams}
Venn diagrams visually represent set operations and relationships.

\section{Conclusion}
This guide introduces the basics of logic, quantifiers, and set theory. Mastery of these topics provides a strong foundation for advanced studies in mathematics, computer science, and related fields.

\end{document}
