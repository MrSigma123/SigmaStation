\documentclass[12pt]{article}
\usepackage{amsmath}
\usepackage{amssymb}
\usepackage{geometry}
\geometry{a4paper, margin=1in}
\setlength{\parindent}{0pt}
\setlength{\parskip}{1em}

\title{Basic Rules of Matrix Manipulation}
\author{}
\date{}

\begin{document}

\maketitle

\section*{Introduction}
Matrices are fundamental tools in mathematics, physics, engineering, and computer science. This document provides an overview of the basic rules of matrix manipulation.

\section*{Matrix Addition and Subtraction}
\begin{itemize}
    \item Matrices can be added or subtracted only if they have the same dimensions.
    \item If \( A = [a_{ij}] \) and \( B = [b_{ij}] \) are two \( m \times n \) matrices, their sum \( C = A + B \) is defined as:
    \[
    C = [c_{ij}], \quad c_{ij} = a_{ij} + b_{ij}.
    \]
\end{itemize}

\section*{Scalar Multiplication}
\begin{itemize}
    \item A matrix \( A = [a_{ij}] \) can be multiplied by a scalar \( k \). The result is:
    \[
    kA = [k \cdot a_{ij}].
    \]
\end{itemize}

\section*{Matrix Multiplication}
\begin{itemize}
    \item The product of two matrices \( A \) and \( B \) is defined only if the number of columns in \( A \) equals the number of rows in \( B \).
    \item If \( A \) is an \( m \times n \) matrix and \( B \) is an \( n \times p \) matrix, the product \( C = AB \) is an \( m \times p \) matrix, where:
    \[
    c_{ij} = \sum_{k=1}^n a_{ik} b_{kj}.
    \]
\end{itemize}

\section*{Transpose of a Matrix}
\begin{itemize}
    \item The transpose of a matrix \( A \), denoted by \( A^\top \), is obtained by interchanging its rows and columns.
    \item If \( A = [a_{ij}] \), then:
    \[
    A^\top = [a_{ji}].
    \]
\end{itemize}

\section*{Identity Matrix}
\begin{itemize}
    \item The identity matrix \( I_n \) is a square matrix of size \( n \times n \) with ones on the diagonal and zeros elsewhere:
    \[
    I_n = 
    \begin{bmatrix}
    1 & 0 & \cdots & 0 \\
    0 & 1 & \cdots & 0 \\
    \vdots & \vdots & \ddots & \vdots \\
    0 & 0 & \cdots & 1
    \end{bmatrix}.
    \]
    \item \( AI_n = I_nA = A \) for any \( n \times n \) matrix \( A \).
\end{itemize}

\section*{Matrix Inversion}
\begin{itemize}
    \item A square matrix \( A \) is invertible (or non-singular) if there exists a matrix \( A^{-1} \) such that:
    \[
    AA^{-1} = A^{-1}A = I_n.
    \]
    \item Only square matrices can have inverses, and not all square matrices are invertible.
\end{itemize}

\section*{Properties of Matrix Operations}
\begin{enumerate}
    \item \textbf{Associativity of Addition:} \( A + (B + C) = (A + B) + C \).
    \item \textbf{Commutativity of Addition:} \( A + B = B + A \).
    \item \textbf{Distributive Property:} \( A(B + C) = AB + AC \) and \( (A + B)C = AC + BC \).
    \item \textbf{Associativity of Multiplication:} \( A(BC) = (AB)C \).
    \item \textbf{Transpose Properties:}
    \[
    (A + B)^\top = A^\top + B^\top, \quad (AB)^\top = B^\top A^\top.
    \]
\end{enumerate}

\section*{Conclusion}
These basic rules of matrix manipulation are foundational for understanding advanced topics in linear algebra. Mastery of these operations will enable you to solve complex problems in various disciplines.

\end{document}
