\documentclass[12pt,a4paper]{article}
\usepackage{amsmath, amssymb, amsthm, geometry, hyperref}
\geometry{margin=1in}

\title{\textbf{Methods of Mathematical Proofs}}
\author{Your Name}
\date{\today}

\begin{document}

\maketitle

\tableofcontents
\newpage

\section{Introduction}
Proofs are the foundation of mathematics. They provide a rigorous way to demonstrate the truth of mathematical statements. In this document, we will explore various proof methods, including direct proofs, proof by contradiction, proof by induction, and proof by contrapositive. Each method will be illustrated with examples of varying difficulty levels: easy, medium, and advanced.

\newpage

\section{Direct Proof}
A direct proof involves starting from the given assumptions and using logical deductions to arrive at the conclusion. This method is straightforward and commonly used in mathematics.

\subsection{Example 1: Easy}
\textbf{Statement:} If $n$ is an even integer, then $n^2$ is also even.

\textbf{Proof:}
\begin{itemize}
    \item Assume $n$ is even. By definition, $n = 2k$ for some integer $k$.
    \item Calculate $n^2$: $n^2 = (2k)^2 = 4k^2 = 2(2k^2)$.
    \item Since $2k^2$ is an integer, $n^2$ is even.\qed
\end{itemize}

\subsection{Example 2: Medium}
\textbf{Statement:} For any two even integers $a$ and $b$, their sum $a+b$ is even.

\textbf{Proof:}
\begin{itemize}
    \item Assume $a$ and $b$ are even. By definition, $a = 2k$ and $b = 2m$ for some integers $k$ and $m$.
    \item Calculate $a+b$: $a+b = 2k + 2m = 2(k+m)$.
    \item Since $k+m$ is an integer, $a+b$ is even.\qed
\end{itemize}

\subsection{Example 3: Advanced}
\textbf{Statement:} For any integers $a$ and $b$, if $a$ and $b$ are both divisible by 4, then $a^2 + b^2$ is divisible by 16.

\textbf{Proof:}
\begin{itemize}
    \item Assume $a$ and $b$ are divisible by 4. By definition, $a = 4k$ and $b = 4m$ for some integers $k$ and $m$.
    \item Calculate $a^2 + b^2$: $a^2 + b^2 = (4k)^2 + (4m)^2 = 16k^2 + 16m^2 = 16(k^2 + m^2)$.
    \item Since $k^2 + m^2$ is an integer, $a^2 + b^2$ is divisible by 16.\qed
\end{itemize}

\newpage

\section{Proof by Contradiction}
Proof by contradiction involves assuming the negation of the statement to be proved and then showing that this assumption leads to a contradiction.

\subsection{Example 1: Easy}
\textbf{Statement:} $\sqrt{2}$ is irrational.

\textbf{Proof:}
\begin{itemize}
    \item Assume the negation: $\sqrt{2}$ is rational. Then $\sqrt{2} = \frac{p}{q}$, where $p$ and $q$ are integers with $\gcd(p, q) = 1$.
    \item Square both sides: $2 = \frac{p^2}{q^2}$, so $p^2 = 2q^2$.
    \item This implies $p^2$ is even, so $p$ is even. Let $p = 2k$. Then $p^2 = 4k^2$.
    \item Substitute: $4k^2 = 2q^2$, so $q^2 = 2k^2$. Thus, $q$ is also even.
    \item Both $p$ and $q$ are even, contradicting $\gcd(p, q) = 1$.\qed
\end{itemize}

\subsection{Example 2: Medium}
\textbf{Statement:} There is no largest prime number.

\textbf{Proof:}
\begin{itemize}
    \item Assume the negation: There is a largest prime number, say $p_n$.
    \item Consider the number $P = p_1p_2\cdots p_n + 1$, where $p_1, p_2, \dots, p_n$ are all the primes up to $p_n$.
    \item $P$ is not divisible by any $p_i$ (remainder is 1).
    \item Hence, $P$ is either prime or divisible by a prime greater than $p_n$, contradicting the assumption.\qed
\end{itemize}

\subsection{Example 3: Advanced}
\textbf{Statement:} The decimal expansion of $\pi$ is non-repeating.

\textbf{Proof:}
\begin{itemize}
    \item Assume the negation: $\pi$ has a repeating decimal expansion, so $\pi$ is rational.
    \item By definition, $\pi = \frac{p}{q}$ for integers $p$ and $q$.
    \item However, $\pi$ is known to be transcendental, which means it is not algebraic (not the root of any polynomial with integer coefficients).
    \item This contradicts the assumption that $\pi$ is rational.\qed
\end{itemize}

\newpage

\section{Proof by Contrapositive}
Proof by contrapositive involves proving a statement by showing that its contrapositive is true. Recall that a statement "if $P$, then $Q$" is logically equivalent to "if not $Q$, then not $P$."

\subsection{Example 1: Easy}
\textbf{Statement:} If $n^2$ is even, then $n$ is even.

\textbf{Proof:}
\begin{itemize}
    \item Contrapositive: If $n$ is odd, then $n^2$ is odd.
    \item Assume $n$ is odd. By definition, $n = 2k+1$ for some integer $k$.
    \item Calculate $n^2$: $n^2 = (2k+1)^2 = 4k^2 + 4k + 1 = 2(2k^2 + 2k) + 1$.
    \item Since $2k^2 + 2k$ is an integer, $n^2$ is odd.\qed
\end{itemize}

\subsection{Example 2: Medium}
\textbf{Statement:} If $a \cdot b = 0$, then $a = 0$ or $b = 0$.

\textbf{Proof:}
\begin{itemize}
    \item Contrapositive: If $a \neq 0$ and $b \neq 0$, then $a \cdot b \neq 0$.
    \item Assume $a \neq 0$ and $b \neq 0$. By definition, both $a$ and $b$ are nonzero numbers.
    \item The product of two nonzero numbers is nonzero: $a \cdot b \neq 0$.\qed
\end{itemize}

\subsection{Example 3: Advanced}
\textbf{Statement:} If a number is not divisible by 3, then its square is not divisible by 3.

\textbf{Proof:}
\begin{itemize}
    \item Contrapositive: If a number's square is divisible by 3, then the number is divisible by 3.
    \item Assume $n^2$ is divisible by 3. Then $n^2 = 3k$ for some integer $k$.
    \item If $n^2$ is divisible by 3, then $n$ must also be divisible by 3 (since 3 is prime).
    \item Thus, if $n^2$ is divisible by 3, then $n$ is divisible by 3.\qed
\end{itemize}

\newpage

\section{Proof by Induction}
Proof by induction is used to prove statements about integers. It consists of a base case and an inductive step.

\subsection{Example 1: Easy}
\textbf{Statement:} $1 + 2 + 3 + \cdots + n = \frac{n(n+1)}{2}$ for all $n \geq 1$.

\textbf{Proof:}
\begin{itemize}
    \item Base case ($n=1$): $1 = \frac{1(1+1)}{2}$. True.
    \item Inductive step: Assume true for $n=k$: $1+2+\cdots+k = \frac{k(k+1)}{2}$.
    \item Prove for $n=k+1$: $1+2+\cdots+k+(k+1) = \frac{k(k+1)}{2} + (k+1)$.
    \item Simplify: $\frac{k(k+1)}{2} + (k+1) = \frac{k(k+1) + 2(k+1)}{2} = \frac{(k+1)(k+2)}{2}$.
    \item True for $n=k+1$. By induction, true for all $n\geq1$.\qed
\end{itemize}

\subsection{Example 2: Medium}
\textbf{Statement:} $2^n > n^2$ for all $n \geq 5$.

\textbf{Proof:}
\begin{itemize}
    \item Base case ($n=5$): $2^5 = 32 > 25 = 5^2$. True.
    \item Inductive step: Assume true for $n=k$: $2^k > k^2$.
    \item Prove for $n=k+1$: $2^{k+1} = 2 \cdot 2^k > 2 \cdot k^2$ (by assumption).
    \item Need $2k^2 > (k+1)^2$:
    \begin{align*}
        2k^2 - (k+1)^2 &= 2k^2 - (k^2 + 2k + 1) \\
        &= k^2 - 2k - 1.
    \end{align*}
    \item For $k\geq 5$, $k^2 - 2k - 1 > 0$. True for $n=k+1$.\qed
\end{itemize}

\subsection{Example 3: Advanced}
\textbf{Statement:} $\sum_{i=1}^n i^3 = \left(\frac{n(n+1)}{2}\right)^2$ for all $n \geq 1$.

\textbf{Proof:}
\begin{itemize}
    \item Base case ($n=1$): $1^3 = \left(\frac{1(1+1)}{2}\right)^2 = 1$. True.
    \item Inductive step: Assume true for $n=k$: $\sum_{i=1}^k i^3 = \left(\frac{k(k+1)}{2}\right)^2$.
    \item Prove for $n=k+1$: $\sum_{i=1}^{k+1} i^3 = \left(\frac{k(k+1)}{2}\right)^2 + (k+1)^3$.
    \item Simplify: $\left(\frac{k(k+1)}{2}\right)^2 + (k+1)^3 = \left(\frac{(k+1)(k+2)}{2}\right)^2$ (details omitted for brevity).
    \item True for $n=k+1$. By induction, true for all $n\geq1$.\qed
\end{itemize}

\newpage

\section{Conclusion}
In this document, we have explored various methods of mathematical proofs, providing clear, step-by-step examples of each method with increasing levels of difficulty. These examples illustrate the power and utility of mathematical reasoning in establishing truths across diverse domains.

\end{document}

