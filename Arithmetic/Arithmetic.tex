\documentclass{article}
\usepackage[utf8]{inputenc}
\usepackage{amsmath}
\usepackage{amssymb}
\usepackage{geometry}
\geometry{a4paper, margin=1in}

\title{Understanding Arithmetic: Basics and Examples}
\author{}
\date{}

\begin{document}

\maketitle

\section*{Introduction}
Arithmetic is the branch of mathematics that deals with the study of numbers and the basic operations performed on them: addition, subtraction, multiplication, and division. These operations are fundamental in mathematics and have applications in everyday life, from simple counting to complex calculations. In this article, we will explore each operation with explanations and examples.

\section{Addition}
Addition is the process of combining two or more numbers to find their total.
\begin{description}
    \item[Notation:] The symbol for addition is "+".
    \item[Example:] $5 + 3 = 8$
\end{description}
\subsection*{Properties of Addition}
\begin{itemize}
    \item \textbf{Commutative Property:} $a + b = b + a$
    \item \textbf{Associative Property:} $(a + b) + c = a + (b + c)$
    \item \textbf{Identity Property:} $a + 0 = a$
\end{itemize}
\textbf{Example:} Verify the commutative property for $2 + 4$:
\[2 + 4 = 6 \quad \text{and} \quad 4 + 2 = 6\]
\textit{Since both results are equal, the property holds.}

\section{Subtraction}
Subtraction is the operation of finding the difference between two numbers.
\begin{description}
    \item[Notation:] The symbol for subtraction is "\(-\)".
    \item[Example:] $10 - 4 = 6$
\end{description}
Unlike addition, subtraction is not commutative or associative.
\textbf{Example:} Calculate $15 - 9$:
\[15 - 9 = 6\]

\section{Multiplication}
Multiplication is the process of adding a number to itself a certain number of times.
\begin{description}
    \item[Notation:] The symbol for multiplication is "\(\times\)" or "\(\cdot\)".
    \item[Example:] $4 \times 3 = 12$
\end{description}
\subsection*{Properties of Multiplication}
\begin{itemize}
    \item \textbf{Commutative Property:} $a \times b = b \times a$
    \item \textbf{Associative Property:} $(a \times b) \times c = a \times (b \times c)$
    \item \textbf{Identity Property:} $a \times 1 = a$
    \item \textbf{Zero Property:} $a \times 0 = 0$
\end{itemize}
\textbf{Example:} Verify the associative property for $2 \times 3 \times 4$:
\[(2 \times 3) \times 4 = 6 \times 4 = 24\]
\[2 \times (3 \times 4) = 2 \times 12 = 24\]
\textit{Since both results are equal, the property holds.}

\section{Division}
Division is the process of distributing a number into equal parts.
\begin{description}
    \item[Notation:] The symbol for division is "\(\div\)" or "\(/\)".
    \item[Example:] $20 \div 4 = 5$
\end{description}
Division is not commutative or associative. Additionally, division by zero is undefined.
\textbf{Example:} Calculate $36 \div 6$:
\[36 \div 6 = 6\]

\section*{Conclusion}
Arithmetic is a fundamental part of mathematics, forming the basis for more advanced topics. Understanding the properties and operations allows for effective problem-solving in both academic and real-world contexts.

\end{document}
