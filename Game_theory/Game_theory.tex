\documentclass[a4paper,12pt]{article}
\usepackage{amsmath}
\usepackage{geometry}
\geometry{margin=1in}

\title{An Introduction to Game Theory and Its Basics}
\author{}
\date{}

\begin{document}

\maketitle

\section*{Introduction}
Game theory is a branch of mathematics that studies strategic interactions between rational decision-makers. It provides a framework for analyzing situations where the outcome for each participant depends on the choices made by others. Although initially developed to model economic behavior, game theory is now widely applied in fields such as political science, biology, computer science, and social sciences.

\section*{Key Concepts in Game Theory}

\subsection*{Players}
A \textit{player} is an individual or entity that makes decisions in a strategic situation. Players can represent individuals, companies, governments, or even algorithms in automated systems.

\subsection*{Strategies}
A \textit{strategy} is a complete plan of action for a player, specifying what they will do in every possible situation they might face during the game. A player’s choice of strategy influences the overall outcome of the game.

\subsection*{Payoffs}
Each player receives a \textit{payoff}, which represents the benefit or utility they gain from the outcome of the game. Payoffs are typically represented as numerical values that the player aims to maximize.

\subsection*{Types of Games}
Games can be categorized in various ways:
\begin{itemize}
    \item \textbf{Cooperative vs. Non-Cooperative}: In cooperative games, players can form binding agreements, whereas in non-cooperative games, they cannot.
    \item \textbf{Zero-Sum vs. Non-Zero-Sum}: In zero-sum games, one player’s gain is exactly equal to another player’s loss. In non-zero-sum games, mutual gains or losses are possible.
    \item \textbf{Simultaneous vs. Sequential}: In simultaneous games, players make decisions at the same time, without knowing the other’s choices. In sequential games, players take turns making decisions.
\end{itemize}

\section*{Conclusion}
Game theory provides a powerful toolkit for understanding complex interactions in competitive and cooperative environments. By analyzing strategies, payoffs, and player behavior, it helps to predict outcomes and guide decision-making in real-world scenarios.

\end{document}
