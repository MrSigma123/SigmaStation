\documentclass[a4paper, 12pt]{article}
\usepackage{amsmath, amssymb, amsthm, enumitem, geometry}
\geometry{margin=1in}

\title{Comprehensive Guide to Series and Convergence Tests}
\author{}
\date{}

\begin{document}

\maketitle

\section*{Introduction to Series}
A series is the sum of the terms of a sequence \( \{a_n\} \):
\[
S = \sum_{n=1}^\infty a_n.
\]
Key questions about a series are whether it converges and under what conditions. This document covers the fundamental concepts, conditions, and various convergence tests.

\section{Necessary Condition for Convergence}
A necessary condition for the convergence of a series \(\sum a_n\) is:
\[
\lim_{n \to \infty} a_n = 0.
\]
If this condition is not met, the series diverges. Note that this condition is not sufficient for convergence.

\subsection*{How to Apply}
\begin{enumerate}
    \item Take the general term \(a_n\) of the series.
    \item Compute \(\lim_{n \to \infty} a_n\).
    \item If \(\lim_{n \to \infty} a_n \neq 0\), the series diverges.
    \item If \(\lim_{n \to \infty} a_n = 0\), further tests are needed to check convergence.
\end{enumerate}

\subsection*{Examples}
\begin{enumerate}
    \item \(a_n = \frac{1}{n}\):
    \[\lim_{n \to \infty} a_n = 0,\]
    but \(\sum \frac{1}{n}\) diverges (harmonic series).
    
    \item \(a_n = \frac{1}{n^2}\):
    \[\lim_{n \to \infty} a_n = 0,\]
    and \(\sum \frac{1}{n^2}\) converges (p-series with \(p > 1\)).

    \item \(a_n = \frac{\sin(n)}{n}\):
    \[\lim_{n \to \infty} a_n = 0,\]
    but additional tests are required to determine convergence.
\end{enumerate}

\section{Absolutely and Conditionally Convergent Series}
\textbf{Absolute convergence:} \(\sum |a_n|\) converges.

\textbf{Conditional convergence:} \(\sum a_n\) converges, but \(\sum |a_n|\) diverges.

\subsection*{How to Apply}
\begin{enumerate}
    \item Compute \(\sum |a_n|\): Replace all terms \(a_n\) with their absolute values.
    \item If \(\sum |a_n|\) converges, the series is absolutely convergent.
    \item If \(\sum a_n\) converges but \(\sum |a_n|\) diverges, the series is conditionally convergent.
\end{enumerate}

\subsection*{Examples}
\begin{enumerate}
    \item \(a_n = \frac{(-1)^n}{n}\):
    \(\sum a_n\) converges conditionally (alternating harmonic series).

    \item \(a_n = \frac{(-1)^n}{n^2}\):
    \(\sum a_n\) converges absolutely (p-series with \(p > 1\)).

    \item \(a_n = \frac{(-1)^n}{\sqrt{n}}\):
    \(\sum a_n\) converges conditionally.
\end{enumerate}

\section{Convergence Tests}

\subsection{Direct Comparison Test}
This test determines the convergence or divergence of a series by comparing it to a known benchmark series. For a series \(\sum a_n\):
\begin{itemize}
    \item If \(0 \leq a_n \leq b_n\) for all \(n\) and \(\sum b_n\) converges, then \(\sum a_n\) also converges.
    \item If \(a_n \geq b_n \geq 0\) for all \(n\) and \(\sum b_n\) diverges, then \(\sum a_n\) also diverges.
\end{itemize}

\subsection*{How to Apply}
\begin{enumerate}
    \item Identify a comparison series \(\sum b_n\) that is simpler and whose convergence is known.
    \item Check that \(a_n \leq b_n\) for all \(n\).
    \item Determine whether \(\sum b_n\) converges or diverges.
    \item Conclude the same for \(\sum a_n\).
\end{enumerate}

\subsection*{Examples}
\begin{enumerate}
    \item \(a_n = \frac{1}{n^2 + 1}\): Compare to \(b_n = \frac{1}{n^2}\):
    \[
    0 \leq \frac{1}{n^2 + 1} \leq \frac{1}{n^2}.
    \]
    Since \(\sum \frac{1}{n^2}\) converges, \(\sum \frac{1}{n^2 + 1}\) also converges.

    \item \(a_n = \frac{1}{n}\): Compare to \(b_n = \frac{1}{\sqrt{n}}\):
    \[
    \frac{1}{n} \geq \frac{1}{\sqrt{n}} > 0.
    \]
    Since \(\sum \frac{1}{\sqrt{n}}\) diverges, \(\sum \frac{1}{n}\) also diverges.

    \item \(a_n = \frac{\ln(n)}{n^2}\): Compare to \(b_n = \frac{1}{n^2}\):
    \[
    \frac{\ln(n)}{n^2} \leq \frac{1}{n^2},
    \]
    and since \(\sum \frac{1}{n^2}\) converges, \(\sum \frac{\ln(n)}{n^2}\) also converges.
\end{enumerate}

\subsection{Ratio Test (d'Alembert's Criterion)}
This test examines the ratio of consecutive terms:
\[
L = \lim_{n \to \infty} \left|\frac{a_{n+1}}{a_n}\right|.
\]
\begin{itemize}
    \item If \(L < 1\), the series converges absolutely.
    \item If \(L > 1\) or \(L = \infty\), the series diverges.
    \item If \(L = 1\), the test is inconclusive.
\end{itemize}

\subsection*{How to Apply}
\begin{enumerate}
    \item Compute \(\frac{a_{n+1}}{a_n}\) for the general term.
    \item Take the limit \(\lim_{n \to \infty} \left|\frac{a_{n+1}}{a_n}\right| = L\).
    \item If \(L < 1\), conclude absolute convergence. If \(L > 1\), conclude divergence.
    \item If \(L = 1\), use another test.
\end{enumerate}

\subsection*{Examples}
\begin{enumerate}
    \item \(a_n = \frac{1}{n!}\):
    \[\lim_{n \to \infty} \frac{a_{n+1}}{a_n} = 0 < 1,\]
    hence the series converges absolutely.

    \item \(a_n = \frac{1}{2^n}\):
    \[\lim_{n \to \infty} \frac{a_{n+1}}{a_n} = \frac{1}{2} < 1,\]
    hence the series converges absolutely.

    \item \(a_n = \frac{n}{n^2 + 1}\):
    \[\lim_{n \to \infty} \frac{a_{n+1}}{a_n} = 1,\]
    test is inconclusive.
\end{enumerate}

\subsection{Alternating Series Test (Leibniz's Criterion)}
This test applies to series of the form \(\sum (-1)^n a_n\), where \(a_n\) are positive terms.
\begin{itemize}
    \item The terms \(|a_n|\) must decrease monotonically.
    \item \(\lim_{n \to \infty} a_n = 0\).
\end{itemize}

\subsection*{How to Apply}
\begin{enumerate}
    \item Verify that \(|a_n|\) is decreasing for all \(n\).
    \item Check that \(\lim_{n \to \infty} a_n = 0\).
    \item If both conditions are met, conclude convergence.
\end{enumerate}

\subsection*{Examples}
\begin{enumerate}
    \item \(a_n = \frac{1}{n}\): Alternating harmonic series.
    \item \(a_n = \frac{1}{n^2}\): Converges absolutely.
    \item \(a_n = \frac{1}{\sqrt{n}}\): Converges conditionally.
\end{enumerate}

\subsection{Root Test (Cauchy’s Criterion)}
This test examines the limit of the \(n\)-th root of the terms of the series:
\[
L = \lim_{n \to \infty} \sqrt[n]{|a_n|}.
\]
\begin{itemize}
    \item If \(L < 1\), the series converges absolutely.
    \item If \(L > 1\), the series diverges.
    \item If \(L = 1\), the test is inconclusive.
\end{itemize}

\subsection*{How to Apply}
\begin{enumerate}
    \item Compute \(\sqrt[n]{|a_n|}\) for the general term.
    \item Take the limit \(\lim_{n \to \infty} \sqrt[n]{|a_n|} = L\).
    \item If \(L < 1\), conclude absolute convergence. If \(L > 1\), conclude divergence.
    \item If \(L = 1\), use another test.
\end{enumerate}

\subsection*{Examples}
\begin{enumerate}
    \item \(a_n = \frac{1}{2^n}\):
    \[
    \lim_{n \to \infty} \sqrt[n]{|a_n|} = \frac{1}{2} < 1,\]
    hence the series converges absolutely.

    \item \(a_n = \frac{1}{n}\):
    \[
    \lim_{n \to \infty} \sqrt[n]{|a_n|} = 1,\]
    test is inconclusive.

    \item \(a_n = \frac{1}{n^{n}}\):
    \[
    \lim_{n \to \infty} \sqrt[n]{|a_n|} = 0 < 1,\]
    hence the series converges absolutely.
\end{enumerate}

\subsection{Raabe’s Test}
This test refines the ratio test by considering:
\[
R = n \left(\frac{a_n}{a_{n+1}} - 1\right).
\]
\begin{itemize}
    \item If \(R > 1\), the series converges absolutely.
    \item If \(R < 1\), the series diverges.
    \item If \(R = 1\), the test is inconclusive.
\end{itemize}

\subsection*{How to Apply}
\begin{enumerate}
    \item Compute \(\frac{a_n}{a_{n+1}} - 1\) for the general term.
    \item Multiply by \(n\) and simplify.
    \item Take the limit \(\lim_{n \to \infty} R\).
    \item If \(R > 1\), conclude absolute convergence. If \(R < 1\), conclude divergence.
\end{enumerate}

\subsection*{Examples}
\begin{enumerate}
    \item \(a_n = \frac{1}{n!}\): Converges absolutely.
    \item \(a_n = \frac{1}{n^2}\): Converges absolutely.
    \item \(a_n = \frac{1}{\ln(n) n}\): Diverges.
\end{enumerate}

\subsection{Cauchy Condensation Test}
This test applies to positive, decreasing sequences \(a_n\):
\[
\sum a_n \text{ converges if and only if } \sum 2^n a_{2^n} \text{ converges.}
\]

\subsection*{How to Apply}
\begin{enumerate}
    \item Identify the sequence \(a_n\) and compute \(2^n a_{2^n}\).
    \item Determine whether the new series \(\sum 2^n a_{2^n}\) converges or diverges.
    \item Conclude the same for \(\sum a_n\).
\end{enumerate}

\subsection*{Examples}
\begin{enumerate}
    \item \(a_n = \frac{1}{n^2}\): Condensation yields \(\sum \frac{1}{2^n}\), which converges.
    \item \(a_n = \frac{1}{n}\): Condensation yields \(\sum 1\), which diverges.
    \item \(a_n = \frac{1}{n \ln(n)}\): Condensation shows divergence.
\end{enumerate}

\subsection{Integral Test}
The integral test compares the series \(\sum a_n\) with the improper integral \(\int_1^\infty f(x) dx\):
\begin{itemize}
    \item If \(f(x)\) is positive, continuous, and decreasing for \(x \geq 1\), then the convergence of \(\sum a_n\) and \(\int_1^\infty f(x)dx\) is the same.
\end{itemize}

\subsection*{How to Apply}
\begin{enumerate}
    \item Define \(f(x) = a_n\) and verify that \(f(x)\) is positive, continuous, and decreasing for \(x \geq 1\).
    \item Compute \(\int_1^\infty f(x) dx\).
    \item If the integral converges, conclude that \(\sum a_n\) converges. If it diverges, conclude that \(\sum a_n\) diverges.
\end{enumerate}

\subsection*{Examples}
\begin{enumerate}
    \item \(a_n = \frac{1}{n^2}\):
    \[\int_1^\infty \frac{1}{x^2} dx = 1,\]
    hence the series converges.

    \item \(a_n = \frac{1}{n}\): Harmonic series diverges.

    \item \(a_n = \frac{1}{n \ln(n)}\): Diverges.
\end{enumerate}

\section{Conclusion}
This document provides a detailed overview of series and convergence tests, with examples ranging from basic to advanced. Mastering these concepts and methods is essential for understanding the behavior of series in mathematical analysis.

\end{document}
