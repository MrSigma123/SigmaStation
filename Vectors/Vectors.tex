\documentclass[12pt]{article}
\usepackage{amsmath, amssymb, graphicx, hyperref}
\usepackage{geometry}
\geometry{a4paper, margin=1in}
\usepackage{tikz}
\usepackage{float}

\title{\textbf{Comprehensive Guide to Vectors and Their Operations}}
\author{}
\date{}

\begin{document}

\maketitle

\tableofcontents

\newpage

\section{Introduction to Vectors}
A \textbf{vector} is a mathematical object that has both magnitude and direction. Vectors are used to represent quantities such as displacement, velocity, force, and acceleration in physics and engineering. Unlike \textbf{scalars}, which only have magnitude (e.g., temperature, mass), vectors describe a physical phenomenon more completely.

\subsection{Representation of Vectors}
A vector is typically represented as:
\begin{itemize}
    \item A directed line segment with an arrow.
    \item In component form: $\mathbf{v} = \langle v_x, v_y, v_z \rangle$ in three-dimensional space.
    \item In unit vector notation: $\mathbf{v} = v_x \mathbf{i} + v_y \mathbf{j} + v_z \mathbf{k}$.
\end{itemize}

\paragraph{Example:} A vector $\mathbf{v}$ in two dimensions can be written as $\mathbf{v} = \langle 3, 4 \rangle$ or $\mathbf{v} = 3\mathbf{i} + 4\mathbf{j}$.

\section{Vector Operations}

\subsection{Scalar Multiplication}
Scalar multiplication involves multiplying a vector by a scalar (a real number). This changes the magnitude of the vector but not its direction (unless the scalar is negative, in which case the direction reverses).

\paragraph{Mathematical Definition:}
\[
\text{If } \mathbf{v} = \langle v_x, v_y, v_z \rangle, \text{ then } k\mathbf{v} = \langle kv_x, kv_y, kv_z \rangle.
\]

\paragraph{Example:}
Let $\mathbf{v} = \langle 2, -1 \rangle$ and $k = 3$. Then:
\[
k\mathbf{v} = 3\langle 2, -1 \rangle = \langle 6, -3 \rangle.
\]

\subsection{Vector Addition and Subtraction}
\begin{itemize}
    \item \textbf{Addition:} Add corresponding components: $\mathbf{u} + \mathbf{v} = \langle u_x + v_x, u_y + v_y, u_z + v_z \rangle$.
    \item \textbf{Subtraction:} Subtract corresponding components: $\mathbf{u} - \mathbf{v} = \langle u_x - v_x, u_y - v_y, u_z - v_z \rangle$.
\end{itemize}

\paragraph{Example:}
If $\mathbf{u} = \langle 1, 2 \rangle$ and $\mathbf{v} = \langle 3, -1 \rangle$:
\[
\mathbf{u} + \mathbf{v} = \langle 1+3, 2+(-1) \rangle = \langle 4, 1 \rangle.
\]
\[
\mathbf{u} - \mathbf{v} = \langle 1-3, 2-(-1) \rangle = \langle -2, 3 \rangle.
\]

\subsection{Dot Product (Scalar Product)}
The dot product of two vectors results in a scalar and measures how much one vector projects onto another.

\paragraph{Mathematical Definition:}
\[
\mathbf{u} \cdot \mathbf{v} = u_xv_x + u_yv_y + u_zv_z.
\]
Alternatively, using magnitudes:
\[
\mathbf{u} \cdot \mathbf{v} = |\mathbf{u}| |\mathbf{v}| \cos \theta,
\]
where $\theta$ is the angle between the vectors.

\paragraph{Example:}
Let $\mathbf{u} = \langle 1, 2, 3 \rangle$ and $\mathbf{v} = \langle 4, -5, 6 \rangle$. Then:
\[
\mathbf{u} \cdot \mathbf{v} = (1)(4) + (2)(-5) + (3)(6) = 4 - 10 + 18 = 12.
\]

\paragraph{Application: Work}
In physics, work is defined as the dot product of force and displacement:
\[
W = \mathbf{F} \cdot \mathbf{d}.
\]
If $\mathbf{F} = \langle 10, 0 \rangle$ N and $\mathbf{d} = \langle 5, 5 \rangle$ m, then:
\[
W = \mathbf{F} \cdot \mathbf{d} = (10)(5) + (0)(5) = 50\, \text{J}.
\]

\subsection{Cross Product (Vector Product)}
The cross product of two vectors results in a vector that is perpendicular to both and has magnitude equal to the area of the parallelogram spanned by the vectors.

\paragraph{Mathematical Definition:}
\[
\mathbf{u} \times \mathbf{v} = \begin{vmatrix}
\mathbf{i} & \mathbf{j} & \mathbf{k} \\
 u_x & u_y & u_z \\
 v_x & v_y & v_z
\end{vmatrix}.
\]

\paragraph{Example:}
Let $\mathbf{u} = \langle 1, 2, 3 \rangle$ and $\mathbf{v} = \langle 4, -5, 6 \rangle$. Then:
\[
\mathbf{u} \times \mathbf{v} = \begin{vmatrix}
\mathbf{i} & \mathbf{j} & \mathbf{k} \\
 1 & 2 & 3 \\
 4 & -5 & 6
\end{vmatrix} = \mathbf{i}(2\cdot6 - 3\cdot(-5)) - \mathbf{j}(1\cdot6 - 3\cdot4) + \mathbf{k}(1\cdot(-5) - 2\cdot4).
\]
\[
\mathbf{u} \times \mathbf{v} = \mathbf{i}(27) - \mathbf{j}(-6) + \mathbf{k}(-13) = 27\mathbf{i} + 6\mathbf{j} - 13\mathbf{k}.
\]

\paragraph{Application: Moment of Inertia}
The moment of inertia is calculated using cross products in rotational dynamics:
\[
\mathbf{\tau} = \mathbf{r} \times \mathbf{F},
\]
where $\mathbf{r}$ is the position vector and $\mathbf{F}$ is the force vector.

\paragraph{Example:}
Let $\mathbf{r} = \langle 3, 0, 0 \rangle$ m and $\mathbf{F} = \langle 0, 5, 0 \rangle$ N. Then:
\[
\mathbf{\tau} = \mathbf{r} \times \mathbf{F} = \begin{vmatrix}
\mathbf{i} & \mathbf{j} & \mathbf{k} \\
 3 & 0 & 0 \\
 0 & 5 & 0
\end{vmatrix} = \mathbf{i}(0) - \mathbf{j}(0) + \mathbf{k}(15) = 15\mathbf{k}.
\]

\section{Conclusion}
Vectors are fundamental in mathematics and physics, allowing us to represent and compute physical phenomena efficiently. Understanding vector operations is key to solving problems in mechanics, electromagnetism, and other fields.

\end{document}
