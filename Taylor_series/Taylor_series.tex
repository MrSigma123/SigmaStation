\documentclass[12pt]{article}
\usepackage{amsmath, amssymb}
\usepackage{geometry}
\geometry{a4paper, margin=1in}

\title{Understanding Taylor Series: Explanation and Examples}
\author{}
\date{}

\begin{document}

\maketitle

\section*{Introduction}
The Taylor series is a fundamental concept in calculus and mathematical analysis. It allows us to approximate functions using polynomials, which are often simpler to handle. This powerful tool has applications in physics, engineering, computer science, and many other fields.

\section*{What is the Taylor Series?}
The Taylor series represents a function as an infinite sum of terms derived from its derivatives at a single point. Mathematically, the Taylor series of a function $f(x)$ centered at $a$ is given by:
\[
T(x) = f(a) + f'(a)(x-a) + \frac{f''(a)}{2!}(x-a)^2 + \frac{f'''(a)}{3!}(x-a)^3 + \dots
\]

This can be written compactly as:
\[
T(x) = \sum_{n=0}^{\infty} \frac{f^{(n)}(a)}{n!}(x-a)^n,
\]

where:
\begin{itemize}
    \item $f^{(n)}(a)$ is the $n$-th derivative of $f$ evaluated at $a$,
    \item $n!$ is the factorial of $n$,
    \item $x$ is the variable,
    \item $a$ is the center of the expansion.
\end{itemize}

\section*{Why Use the Taylor Series?}
Functions can sometimes be difficult to evaluate directly. The Taylor series provides an approximation that is easier to compute, especially for values of $x$ close to $a$. For instance:
\begin{itemize}
    \item Polynomial approximations can simplify complex functions in numerical methods.
    \item Engineers often use Taylor series to linearize non-linear systems near an equilibrium point.
\end{itemize}

\section*{Examples of Taylor Series}

\subsection*{1. Taylor Series for $e^x$ (centered at $a=0$)}
The exponential function $e^x$ has the property that all its derivatives are equal to $e^x$. At $x=0$, we have $e^0 = 1$. Thus:
\[
T(x) = 1 + x + \frac{x^2}{2!} + \frac{x^3}{3!} + \dots
\]

This gives the infinite series:
\[
e^x = \sum_{n=0}^{\infty} \frac{x^n}{n!}.
\]

\subsection*{2. Taylor Series for $\sin(x)$ (centered at $a=0$)}
The derivatives of $\sin(x)$ follow a repeating pattern:
\[
\sin(x) \to \cos(x) \to -\sin(x) \to -\cos(x) \to \sin(x).
\]

At $x=0$, we find:
\[
\sin(0) = 0, \quad \cos(0) = 1.
\]

Substituting these values:
\[
T(x) = x - \frac{x^3}{3!} + \frac{x^5}{5!} - \frac{x^7}{7!} + \dots
\]

The series is:
\[
\sin(x) = \sum_{n=0}^{\infty} (-1)^n \frac{x^{2n+1}}{(2n+1)!}.
\]

\subsection*{3. Taylor Series for $\ln(1+x)$ (centered at $a=0$)}
For $\ln(1+x)$, the derivatives are:
\[
\ln'(1+x) = \frac{1}{1+x}, \quad \ln''(1+x) = -\frac{1}{(1+x)^2}, \dots
\]

At $x=0$:
\[
\ln(1+0) = 0, \quad \ln'(1+0) = 1, \quad \ln''(1+0) = -1, \dots
\]

The series is:
\[
\ln(1+x) = x - \frac{x^2}{2} + \frac{x^3}{3} - \frac{x^4}{4} + \dots
\]

\section*{Convergence of the Taylor Series}
The Taylor series does not always converge to the function it represents. The interval of convergence depends on the function and the center $a$. For example:
\begin{itemize}
    \item $e^x$ converges for all $x$.
    \item $\ln(1+x)$ converges only for $-1 < x \leq 1$.
\end{itemize}

\section*{Practical Applications}
\begin{enumerate}
    \item \textbf{Physics:} Approximating complex waveforms using sine and cosine Taylor series expansions.
    \item \textbf{Engineering:} Linearizing control systems and circuits.
    \item \textbf{Computer Science:} Implementing efficient algorithms for function evaluation.
\end{enumerate}

\section*{Summary}
The Taylor series is a versatile tool that bridges the gap between complex functions and polynomial approximations. By understanding its formulation, convergence, and applications, we gain a deeper insight into its utility in various scientific and engineering domains.

\section*{Exercises}
\begin{enumerate}
    \item Find the Taylor series for $\cos(x)$ centered at $a=0$.
    \item Determine the radius of convergence for the Taylor series of $\ln(1+x)$.
    \item Approximate $e^{0.1}$ using the first four terms of its Taylor series.
\end{enumerate}

\end{document}

