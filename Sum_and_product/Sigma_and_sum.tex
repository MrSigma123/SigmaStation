\documentclass{article}
\usepackage{amsmath}
\usepackage{amssymb}
\usepackage{geometry}
\geometry{a4paper, margin=1in}

\title{Sums (\(\Sigma\)) and Products (\(\Pi\)) in Mathematics}
\author{}
\date{}

\begin{document}

\maketitle

\section*{Introduction}
In mathematics, summation and product notation are powerful tools used to represent the addition and multiplication of sequences of terms. The sigma (\(\Sigma\)) notation is used to denote sums, while the pi (\(\Pi\)) notation is used to denote products. This document provides a thorough explanation of both notations, their properties, and examples.

\section{Summation (\(\Sigma\) Notation)}
Summation is the process of adding a sequence of numbers. The sigma notation provides a concise way to represent such sums. The general form of a sum using sigma notation is given by:
\[
\sum_{k = m}^{n} a_k
\]
where:
\begin{itemize}
    \item \(k\) is the index of summation,
    \item \(m\) is the lower bound of the sum,
    \item \(n\) is the upper bound of the sum,
    \item \(a_k\) is the general term of the sequence.
\end{itemize}

\subsection{Properties of Summation}
\begin{enumerate}
    \item \textbf{Linearity:}
    \[
    \sum_{k=m}^{n} (a_k + b_k) = \sum_{k=m}^{n} a_k + \sum_{k=m}^{n} b_k
    \]
    \[
    \sum_{k=m}^{n} c \cdot a_k = c \cdot \sum_{k=m}^{n} a_k \quad \text{(for any constant } c\text{)}
    \]
    
    \item \textbf{Splitting the Sum:}
    If \(p\) is an integer such that \(m \leq p \leq n\), then:
    \[
    \sum_{k=m}^{n} a_k = \sum_{k=m}^{p} a_k + \sum_{k=p+1}^{n} a_k
    \]

    \item \textbf{Summation of Constants:}
    \[
    \sum_{k=m}^{n} c = c \cdot (n - m + 1)
    \]
    where \(c\) is a constant.
\end{enumerate}

\subsection{Examples}
\begin{enumerate}
    \item \textbf{Sum of the First \(n\) Natural Numbers:}
    \[
    \sum_{k=1}^{n} k = \frac{n(n+1)}{2}
    \]
    
    \item \textbf{Sum of the First \(n\) Squares:}
    \[
    \sum_{k=1}^{n} k^2 = \frac{n(n+1)(2n+1)}{6}
    \]
    
    \item \textbf{Sum of the First \(n\) Cubes:}
    \[
    \sum_{k=1}^{n} k^3 = \left(\frac{n(n+1)}{2}\right)^2
    \]
\end{enumerate}

\section{Product (\(\Pi\) Notation)}
The product notation (pi notation) represents the product of a sequence of terms. The general form of a product using pi notation is given by:
\[
\prod_{k = m}^{n} a_k
\]
where:
\begin{itemize}
    \item \(k\) is the index of multiplication,
    \item \(m\) is the lower bound of the product,
    \item \(n\) is the upper bound of the product,
    \item \(a_k\) is the general term of the sequence.
\end{itemize}

\subsection{Properties of Products}
\begin{enumerate}
    \item \textbf{Product of Powers:}
    \[
    \prod_{k=m}^{n} a^b = a^{\sum_{k=m}^{n} b}
    \]
    
    \item \textbf{Product of a Constant:}
    If \(c\) is a constant, then:
    \[
    \prod_{k=m}^{n} c = c^{n-m+1}
    \]
    
    \item \textbf{Multiplicative Splitting:}
    If \(p\) is an integer such that \(m \leq p \leq n\), then:
    \[
    \prod_{k=m}^{n} a_k = \left(\prod_{k=m}^{p} a_k\right) \cdot \left(\prod_{k=p+1}^{n} a_k\right)
    \]
\end{enumerate}

\subsection{Examples}
\begin{enumerate}
    \item \textbf{Product of the First \(n\) Natural Numbers (Factorial):}
    \[
    \prod_{k=1}^{n} k = n! \quad \text{("n factorial")}
    \]
    
    \item \textbf{Product of Powers of 2:}
    \[
    \prod_{k=0}^{n} 2 = 2^{n+1}
    \]
    
    \item \textbf{Product of an Arithmetic Sequence:}
    \[
    \prod_{k=0}^{n} (a + kd)
    \]
    represents the product of terms in an arithmetic progression.
\end{enumerate}

\section*{Conclusion}
The sigma (\(\Sigma\)) and pi (\(\Pi\)) notations provide concise and efficient ways to represent sums and products in mathematics. Their properties and applications are fundamental in many areas of mathematics, including algebra, calculus, and number theory. Mastery of these notations allows for better understanding and manipulation of complex mathematical expressions.

\end{document}
